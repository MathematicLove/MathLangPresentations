\documentclass{beamer}
\usepackage[T1,T2A]{fontenc}  
\usepackage[utf8]{inputenc}  
\usepackage[english,russian]{babel}  
\usepackage{amsmath}  
\usepackage{graphicx}  
\usepackage{listings}   
\usepackage{verbatim}
\usepackage{xcolor}  

\usetheme{Madrid} 
\definecolor{mygreen}{RGB}{50, 50, 220} 

\setbeamercolor{structure}{fg=mygreen} 
\setbeamercolor{palette primary}{fg=white, bg=mygreen} 
\setbeamercolor{titlelike}{fg=white, bg=mygreen} 
\setbeamercolor{block title}{fg=white, bg=mygreen} 

\title[Философия программирования]{Путь в IT и нужна ли математика в программировании сегодня?}
\author{Салимли Айзек}
\institute{MathLang}
\date{\today}
\newenvironment{rusdefinition}[1][Определение]{
    \begin{block}{#1}
}{\end{block}}
\newenvironment{rexample}[1][Пример]{\begin{exampleblock}{#1}}{\end{exampleblock}}

\begin{document}

\begin{frame}
    \titlepage
\end{frame}

\lstset{
    language=Haskell,
    basicstyle=\ttfamily\small,
    inputencoding=utf8,  
    extendedchars=true,
    literate={�}{{\bfseries?}}1 {^^ff}{{\bfseries?}}1,
    breaklines=true,
    breakatwhitespace=true,
    tabsize=2,
    frame=single
}

\AtBeginSection[]{
    \begin{frame}{Содержание}
        \tableofcontents[currentsection]
    \end{frame}
}

\begin{section}{О себе}
    \begin{frame}{О себе}
        \begin{itemize}
            \item Айзек
            \item 22 года
            \item ИКНК, математика и компьютерные науки
            \item 1/2 года работы backend разработчиком \textcolor{olive}{Spring framework} \textcolor{red}{Java}
            \item Год работы с LLM и AI \textcolor{blue}{TheBloomsBridge} (Python)
            
        \end{itemize}
    \end{frame}
\end{section}

\begin{section}{Введение (дополнительное)}
    \begin{frame}{Введение (дополнительное)}
        \begin{itemize}
            \item Все есть абстракция 
            \item Формализм и программирование 
            \item Тезис Черча-Тьюринга \textbf{(???)} 
                \begin{itemize}
                    \item Любая вычислимая функция может быть вычислена на машине Тьюринга в виде алгоритма
                \end{itemize}
            \item Почему сейчас математика не так важна?
            \item \textcolor{red}{Математика - это обобщение}
        \end{itemize}
    \end{frame}
\end{section}

\subsection{Как появились компьютеры и что это?}
    \subsubsection{Машина Тьюринга}
    \begin{frame}{Машина Тьюринга}
        \begin{figure}[H]
            \centering
            \includegraphics[width=0.8\textwidth]{images/turing_machine.png}
            \caption{Машина Тьюринга}
        \end{figure}
    \end{frame}

\subsubsection{Автомат с памятью}
\begin{frame}{Автомат с памятью}
    \begin{figure}[H]
        \centering
        \includegraphics[width=0.8\textwidth]{images/turing_machine_with_memory.png}
        \caption{Автомат с памятью}
    \end{figure}
\end{frame}

\begin{section}{Введение (основное)}
\begin{frame}{Введение (основное)}
    Математика - это обобщение
    \begin{itemize}
        \item Математический анализ 
            \begin{itemize}
                \item Пределы
                \item Производные
                \item Интегралы
                \item Дифференциальные уравнения
                \item Ряды Фурье
                \item Ряды Тейлора
            \end{itemize}
        \item Линейная алгебра 
            \begin{itemize}
                \item Матрицы
                \item Определители
                \item Собственные значения и векторы
                \item Системы линейных уравнений
                \item Векторные пространства
                \item Евклидовы пространства
            \end{itemize}
        \item Дискретная математика 
            \begin{itemize}
                \item Теория множеств
                \item Теория графов
            \end{itemize}
    \end{itemize}
\end{frame}

\begin{frame}{Введение}
    \begin{itemize}
        \item Теория вероятностей и математическая статистика 
            \begin{itemize}
                \item Классическая теория вероятностей
                \item Случайные величины
                \item Случайные вектора
                \item Регрессионный анализ
                \item Статистические методы
            \end{itemize}
        \item Теория алгоритмов 
            \begin{itemize}
                \item Алгоритмы сортировки
                \item Алгоритмы поиска
                \item Алгоритмы на графах
                \item Алгоритмы на деревьях
                \item Алгоритмы на матрицах
            \end{itemize}
        \item Конечные автоматы 
            \begin{itemize}
                \item НДКА
                \item ДКА
                \item Машина Тьюринга
                \item Нормальные алгоритмы Маркова
            \end{itemize}
    \end{itemize}
    Что выбрать и что нужно ?
\end{frame}
\end{section}

\begin{section}{Профессии в IT}
    \begin{frame}{Профессии в IT}
        \begin{itemize}
            \item Программист \begin{itemize}
                \item Backend  
                \item Frontend  
                \item Fullstack 
                \item Mobile
                \item Web 
                \item Game 
                \item Quantum \textbf{(физика и математика обязательны)}
            \end{itemize}
            \item Системный архитектор \begin{itemize}
                \item Архитектура баз данных
                \item Архитектура сетей
                \item Архитектура систем
            \end{itemize}
            \item DevOps \begin{itemize}
                \item DevOps Engineer
                \item DevOps Architect
                \item DevOps Manager
                \item DevOps Analyst
                \item DevOps Developer 
            \end{itemize}
            \end{itemize}
        \end{frame}
        \begin{frame}{Профессии в IT}
            \begin{itemize}
            \item Machine Learning Engineer \begin{itemize}
                \item Создание моделей \textbf{(математика и статистика обязательны)}
                \item Интеграция готовых моделей \textbf{(математика и статистика обязательны)}
            \end{itemize}
            \item Data Engineer / Scientist  \textbf{(математика и статистика обязательны)}
            \item Data Analyst  \textbf{(математика и статистика обязательны)}
            \item Тестировщик 
            \item Сис. админ. 
            \item Компиляторы и формальные языки \textbf{(Дискретная математика и теория алгоритмов обязательны)}
            \item Криптография \textbf{(Дискретная математика обязательна)}
        \end{itemize}
    \end{frame}

    \begin{section}{Математика в IT}
        \begin{frame}{Математика в IT}
            \begin{itemize}
                \item LLM - \textbf{(большие языковые модели)}
                \item BlockChain - \textbf{(создание блокчейнов и криптовалют)}
                \item Quantum - \textbf{(создание квантовых компьютеров)}
                \item Machine Learning - \textbf{(обучение моделей на основе данных)}
                \item Data Science - \textbf{(обработка и анализ данных)}
                \item Computer Vision - \textbf{(Автоматизация обработки изображений и видео)} 
                \item Компиляторы и формальные языки - \textbf{(создание языков программирования, компиляторов, интерпретаторов)}
            \end{itemize}
        \end{frame}
    \end{section}

    \begin{section}{Какой язык выбрать?}
        \begin{frame}{Какой язык выбрать?}
            \begin{itemize}
                \item \textcolor{blue}{Python} 
                \item \textcolor{red}{Java}
                \item \textcolor{olive}{C и C++}
                \item \textcolor{magenta}{C\#}
                \item \textcolor{orange}{JavaScript}
                \item \textcolor{purple}{Go}
                \item \textcolor{brown}{Rust}
                \item \textcolor{pink}{Haskell}
                \item \textcolor{gray}{Q\#}
                \item \textcolor{cyan}{Ruby}
                \item \textcolor{teal}{Swift}
            \end{itemize}
        \end{frame}
    \end{section}

\begin{section}{Современный порог в IT}
    \begin{frame}{Современный порог в IT}
        \begin{itemize}
            \item \textcolor{red}{\textbf{Язык - это инструмент}} 
            \item Тестирование (хотя бы базовые знания)
            \item Понимание принципов \begin{itemize}
                \item \textcolor{red}{ООП}
                \item SOLID
                \item DRY
            \end{itemize}
        \end{itemize}
            \begin{itemize}
                \item \textcolor{red}{Базы данных (SQL, NoSQL, Cache)}
                \item Функциональное программирование
                \item Процедурное программирование
                \item \textcolor{red}{Сети ЭВМ (протоколы передачи данных)}
                \item \textcolor{red}{Архитектура компьютера}
                \item UNIX Системы
                \item REST
                \item Docker, k8s
                \item CI/CD
                \item \textcolor{red}{Git}
                \item \textcolor{red}{Фреймворки}
        \end{itemize}
    \end{frame}
\end{section}





\end{section}

\begin{frame}{}
    \centering
    \Large Спасибо за внимание!
    
    \vspace{1cm}
    \small \textcolor{red}{Пишите вопросы в комментариях!!!}
\end{frame}

\end{document}