\documentclass{beamer}
\usepackage[T1,T2A]{fontenc}
\usepackage[utf8]{inputenc}
\usepackage[english,russian]{babel}
\usepackage{amsmath}
\usepackage{graphicx}
\usepackage{listings}
\usepackage{xcolor}

\usetheme{Madrid}
\definecolor{mygreen}{RGB}{0, 128, 0} 

\setbeamercolor{structure}{fg=mygreen} 
\setbeamercolor{palette primary}{fg=white, bg=mygreen} 
\setbeamercolor{titlelike}{fg=white, bg=mygreen} 
\setbeamercolor{block title}{fg=white, bg=mygreen} 

\title[МОП]{Краткая теория множеств}
\author{Салимли Айзек}
\institute{MathLang}
\date{\today}
\newenvironment{rusdefinition}[1][Определение]{
    \begin{block}{#1}
}{\end{block}}
\newenvironment{rexample}[1][Пример]{\begin{exampleblock}{#1}}{\end{exampleblock}}

\lstset{
    language=Haskell,
    basicstyle=\ttfamily\small,
    inputencoding=utf8,
    extendedchars=true,
    literate={�}{{\bfseries?}}1 {^^ff}{{\bfseries?}}1,
    breaklines=true,
    breakatwhitespace=true,
    tabsize=2,
    frame=single
}

\begin{document}

\begin{frame}
    \titlepage
\end{frame}

\AtBeginSection[]{
    \begin{frame}{Содержание}
        \tableofcontents[currentsection]
    \end{frame}
}

\section{Универсум}
\begin{frame}{Универсум}
\begin{rusdefinition}{Универсум}
Универсум $\mathbb{U}$ - множество, содержащее все возможные множества рассматриваемой теории:
\[ \mathbb{U} = \{x\ |\ x \text{ - множество}\} \]
\end{rusdefinition}

\begin{rexample}{Примеры универсумов}
\begin{itemize}
\item В теории чисел: $\mathbb{U} = \mathbb{Z}$
\item В теории множеств: $\mathbb{U}$ содержит все множества
\end{itemize}
\end{rexample}

\begin{alertblock}{Проблема}
Наивное определение универсума приводит к парадоксам, таким как \textbf{парадокс Рассела}
\end{alertblock}
\end{frame}

\subsection{Парадокс Рассела}
\begin{frame}{Парадокс Рассела}
\begin{rusdefinition}{Формулировка}
Рассмотрим множество $R$ всех множеств, которые \textbf{не} содержат себя в качестве элемента:
\[ R = \{x\ |\ x \notin x\} \]
\end{rusdefinition}

\begin{rexample}{Иллюстрация}
\begin{itemize}
\item Множество всех книг $\notin$ самому себе
\item Множество всех множеств $\in$ самому себе
\end{itemize}
\end{rexample}

\begin{block}{Парадокс}
Если $R \in R$, то по определению $R \notin R$.\\
Если $R \notin R$, то по определению $R \in R$.\\
Получаем противоречие.
\end{block}
\end{frame}

\begin{frame}{Разрешение парадокса}
\begin{rusdefinition}{Аксиоматическое решение}
Теория множеств Цермело-Френкеля (ZFC) запрещает:
\begin{itemize}
\item Создание множества всех множеств
\item Использование неограниченного принципа свертки
\end{itemize}
\end{rusdefinition}

\begin{rexample}{Альтернативы}
\begin{itemize}
\item Теория типов (Рассел)
\item NBG-теория (классы вместо множеств)
\end{itemize}
\end{rexample}

\begin{block}{Значение}
Парадокс показал необходимость строгой аксиоматизации теории множеств
\end{block}
\end{frame}

\section{Операции над множествами}

\begin{frame}{Операции над множествами}
Обычно говорят о бинарных и унарных операциях над множествами.
Пусть даны множестав X, Y.
\begin{itemize}
    \item Дополнение $\bar{X}$
    \item Объединение $X \cup Y$
    \item Пересечение $X \cap Y$
    \item Разность $X \setminus Y$
    \item Симметрическая разность $X \oplus Y$
\end{itemize}
\end{frame}

\subsection{Дополнение множесвта}
\begin{frame}{Дополнение множества}
    \begin{block}{Определение}
    Дополнение $\overline{X}$ - множество элементов, \textbf{не} принадлежащих $X$ в рамках универсального множества $U$.
    \end{block}
    \begin{exampleblock}{Пример}
    Пусть $U = \{1,2,3,4,5\}$, $X = \{1,2\}$\\
    Тогда $\overline{X} = \{3,4,5\}$
    \end{exampleblock}
\end{frame}

\subsection{Свойства дополнения}
\begin{frame}{Свойства дополнения}
    \begin{block}{Основные свойства}
    \begin{itemize}
        \item $\overline{\overline{X}} = X$ (инволютивность)
        \item $X \cap \overline{X} = \varnothing$ (непересекаемость)
        \item $X \cup \overline{X} = U$ (полнота)
    \end{itemize}
    \end{block}
    \begin{alertblock}{Важное следствие}
    Законы де Моргана:
    \begin{itemize}
        \item $\overline{A \cup B} = \overline{A} \cap \overline{B}$
        \item $\overline{A \cap B} = \overline{A} \cup \overline{B}$
    \end{itemize}
    \end{alertblock}
\end{frame}

\subsection{Объединение множеств}
\begin{frame}{Объединение множеств}
    \begin{block}{Определение}
        $X \cup Y$ - множество всех элементов, принадлежащих \textbf{хотя бы одному} из множеств $X$ или $Y$.
    \end{block}
    \begin{exampleblock}{Пример}
        $X = \{1,2,3\},\ Y = \{3,4,5\}$ \\
        $X \cup Y = \{1,2,3,4,5\}$
    \end{exampleblock}
    \begin{block}{Свойства объединения}
        \begin{itemize}
            \item Коммутативность: $X \cup Y = Y \cup X$
            \item Ассоциативность: $(X \cup Y) \cup Z = X \cup (Y \cup Z)$
        \end{itemize}
    \end{block}
\end{frame}

\section{Пересечение множеств}
\begin{frame}{Пересечение множеств}
    \begin{block}{Определение}
        $X \cap Y$ — множество элементов, принадлежащих \textbf{одновременно} и $X$, и $Y$.
    \end{block}
    \begin{exampleblock}{Пример}
        $X = \{1,2,3\},\ Y = \{2,3,4\}$ \\
        $X \cap Y = \{2,3\}$
    \end{exampleblock}
    \begin{block}{Свойства пересечения}
        \begin{itemize}
        \item $X \cap \varnothing = \varnothing$
        \item Дистрибутивность: $X \cap (Y \cup Z) = (X \cap Y) \cup (X \cap Z)$
        \end{itemize}
    \end{block}
\end{frame}

\section{Разность множеств}
\begin{frame}{Разность множеств}
    \begin{block}{Определение}
        Разность $X \setminus Y$ — множество элементов, принадлежащих $X$, но \textbf{не} принадлежащих $Y$.
    \end{block}
    \begin{exampleblock}{Пример}
        $X = \{1,2,3,4\},\ Y = \{3,4,5\}$ \\
        $X \setminus Y = \{1,2\}$
    \end{exampleblock}
    \begin{block}{Ключевые свойства}
        \begin{itemize}
        \item Не коммутативна: $X \setminus Y \neq Y \setminus X$
        \item Связь с дополнением: $X \setminus Y = X \cap \overline{Y}$
        \end{itemize}
    \end{block}
\end{frame}

\section{Симметрическая разность}
\begin{frame}{Симметрическая разность}
    \begin{block}{Определение}
        Симметрическая разность $X \oplus Y = (X \setminus Y) \cup (Y \setminus X)$ — множество элементов, принадлежащих \textbf{ровно одному} из множеств $X$ или $Y$.
    \end{block}
    \begin{exampleblock}{Наглядный пример}
        \begin{itemize}
            \item $X = \{1,2,3\}$
            \item $Y = \{3,4,5\}$
            \item $X \oplus Y = \{1,2,4,5\}$
        \end{itemize}
    \end{exampleblock}
    \begin{block}{Характеристики}
        \begin{itemize}
            \item \textbf{Коммутативность}: $X \oplus Y = Y \oplus X$
            \item \textbf{Самообратимость}: $X \oplus X = \varnothing$
            \item \textbf{Ассоциативность}: $(X \oplus Y) \oplus Z = X \oplus (Y \oplus Z)$
        \end{itemize}
    \end{block}
\end{frame}

\begin{frame}{}
    \centering
    \Large Спасибо за внимание!
    
    \vspace{1cm}
    \small \textcolor{red}{Пишите вопросы в комментариях!!!}
\end{frame}

\end{document}