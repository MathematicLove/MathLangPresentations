\documentclass{beamer}
\usepackage[T2A]{fontenc} 
\usepackage[utf8]{inputenc} 
\usepackage[english,russian]{babel} 
\usepackage{amsmath}  
\usepackage{graphicx}  

\usetheme{Madrid}  
\definecolor{mygreen}{RGB}{0, 128, 0} 

\setbeamercolor{structure}{fg=mygreen} 
\setbeamercolor{palette primary}{fg=white, bg=mygreen} 
\setbeamercolor{titlelike}{fg=white, bg=mygreen} 
\setbeamercolor{block title}{fg=white, bg=mygreen} 

\title[МОП]{Краткая теория множеств}
\author{Салимли Айзек}
\institute{MathLang}
\date{\today}
\newenvironment{rusdefinition}[1][Определение]{
    \begin{block}{#1}
}{\end{block}}
\newenvironment{rexample}[1][Пример]{\begin{exampleblock}{#1}}{\end{exampleblock}}

\begin{document}

\begin{frame}
    \titlepage
\end{frame}

\AtBeginSection[]{
    \begin{frame}{Содержание}
        \tableofcontents[currentsection]
    \end{frame}
}

\section{Множества}

\begin{frame}{Элементы множеств}
    \begin{rusdefinition}
        это совокупность различных объектов, объединённых по какому-либо признаку.
        Эти объекты называются элементами множества.
    \end{rusdefinition}
    \begin{rexample}

    Пусть дано множество X, такое что:
    \[
        \forall x_i \in X, \space i \in [0, 1, \dots, n], n \in \mathbb{N}
    \]
\end{rexample}
    \begin{itemize}
        \item Элементы множества - называется $x_i$, где i - позиция элемента;
        \item $\mathbb{N}$ - множество натуральных чисел: 1, 2, 3, ..., +$\infty$;
        \item $\forall$ - предикат обозначающий: Для любых/любого/всех. 
    \end{itemize}
\end{frame}

\section{Отношения}

\begin{frame}{Отношения}
    Пусть даны два множества $X$. Отношением $R$ называется подмножество декартова произведения:
    \[
        R \subseteq X \times X
    \]
    Образуя пары $(x_i, x_j)$, которые связаны между собой каким-либо отношением R.
    Основные типы отношений (для $R \subseteq X \times X $):
    \begin{enumerate}
        \item \textbf{Рефлексивность}: $\forall x \in X \colon x R x$;
        \item \textbf{Симметричность}: $\forall x, y \in X \colon x R y \implies y R x$;
        \item \textbf{Транзитивность}: $\forall x, y, z \in X \colon x R y \land y R z \implies x R z$.
    \end{enumerate}
    \textbf{\textcolor{red}{Коньюнкция - $\land$ тоже самое что \&\& (логическое и)}}
\end{frame}

\begin{frame}{Отношения}
    \begin{rexample}    
    \[ 12/4 = 6/2 \] 
    \[ 10 \cdot 5 = 5 \cdot 10 \]
    \[ 3 > 2 \texttt{ } \&\& \texttt{ } 3 > 1 \rightarrow 3 > 1 \] 
\end{rexample}
То есть:
\begin{enumerate}
    \item Рефлексивность - когда операция над элементом множества дает этот же элемент;
    \item Симметрия - когда операция над элементами множества равны при их перестановки;
    \item Транзитивность - достижимость начального элемента до n-го элемента через элемент "посредник".
\end{enumerate}
\end{frame}

\section{Булеан}

\begin{frame}{Булеан}
    Пусть дано множество $X$. 

    \begin{rusdefinition}
        Подмножество - это набор элементов, содержащихся в множестве $X$. 
    \end{rusdefinition}

    Например, если $X = \{1,2,3,4,5,6,7,8,9,0\}$, то множество всех четных чисел 
    $X_{\text{even}} = \{2,4,6,8\}$ является подмножеством $X$:
    \[ X_{\text{even}} \subset X \]
    где $\subset$ - знак подмножества.

    \begin{rusdefinition}
        Булеан (множество всех подмножеств) множества $X$ обозначается как $\mathcal{P}(X)$ или $2^X$.
    \end{rusdefinition}
\end{frame}

\begin{frame}{Булеан}
    Мощность булеана вычисляется по формуле:
    \[ |\mathcal{P}(X)| = 2^{|X|} \]
    где $|X|$ - мощность (количество элементов) множества $X$.

    \alert{Важно! Пустое множество $\emptyset$ всегда является элементом булеана.}
\begin{rexample}    
        Для $X = \{a,b,c\}$:
        \[ |\mathcal{P}(X)| = 2^3 = 8 \]
        Булеан включает:
        \begin{itemize}
            \item $\emptyset$ - 1 подмножество
            \item $\{a\}, \{b\}, \{c\}$ - 3 подмножества
            \item $\{a,b\}, \{b,c\}, \{b,c\}$ - 3 подмножества 
            \item $\{a,b,c\}$ - 1 подмножество
        \end{itemize}
    \end{rexample}
\end{frame}

\begin{frame}{}
    \centering
    \Large Спасибо за внимание!
    
    \vspace{1cm}
    \small \textcolor{red}{Пишите вопросы в комментариях!!!}
\end{frame}

\end{document}