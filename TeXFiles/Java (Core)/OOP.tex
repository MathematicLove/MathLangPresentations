\documentclass{beamer}
\usepackage[T1,T2A]{fontenc}
\usepackage[utf8]{inputenc}
\usepackage[english,russian]{babel}
\usepackage{amsmath}
\usepackage{graphicx}
\usepackage{listings}
\usepackage{xcolor}

\usetheme{Madrid}
\definecolor{mygreen}{RGB}{250, 110, 0} 

\setbeamercolor{structure}{fg=mygreen} 
\setbeamercolor{palette primary}{fg=white, bg=mygreen} 
\setbeamercolor{titlelike}{fg=white, bg=mygreen} 
\setbeamercolor{block title}{fg=white, bg=mygreen} 

\title[Java Core]{Начало}
\author{Салимли Айзек}
\institute{MathLang}
\date{\today}
\newenvironment{rusdefinition}[1][Определение]{
    \begin{block}{#1}
}{\end{block}}
\newenvironment{rexample}[1][Пример]{\begin{exampleblock}{#1}}{\end{exampleblock}}
\newenvironment{forMain}[1][Файл Main.java]{\begin{exampleblock}{#1}}{\end{exampleblock}}

\lstset{
    language=Haskell,
    basicstyle=\ttfamily\small,
    inputencoding=utf8,
    extendedchars=true,
    literate={�}{{\bfseries?}}1 {^^ff}{{\bfseries?}}1,
    breaklines=true,
    breakatwhitespace=true,
    tabsize=2,
    frame=single
}

\begin{document}

\begin{frame}
    \titlepage
\end{frame}

\AtBeginSection[]{
    \begin{frame}{Содержание}
        \tableofcontents[currentsection]
    \end{frame}
}

\section{Смысл Java}
\begin{frame}{Смысл Java}
    \begin{itemize}
        \item \textbf{А зачем?}
        \begin{itemize}
            \item Write Once Run Anywhere (WORA)
            \item Переносимость между устройствами (JVM)
            \item Безопасность и надёжность (vs C++)
        \end{itemize}
        
        \item \textbf{А чем лучше друих языков?}
        \begin{itemize}
            \item Автоматическое управление памятью (GC)
            \item Богатая стандартная библиотека
            \item Мультиплатформенность
            \item Легкая реализация многопоточности (IMHO)
        \end{itemize}
        
        \item \textbf{Философия}
        \begin{itemize}
            \item Объектно-ориентированный подход (ООП)
            \item Читаемость кода
            \item Жёсткая типизация + проверки на этапе компиляции
        \end{itemize}
    \end{itemize}
\end{frame}

\section{Что это за слова...}
\begin{frame}{Что это за слова...}
    \begin{itemize}
        \item \textcolor[rgb]{0.0, 0.5, 0.0}{\textbf{JVM (Java Virtual Machine)}} – виртуальная машина, выполняет байткод Java.
        \item \textcolor[rgb]{0.0, 0.5, 0.0}{\textbf{JDK (Java Development Kit)}} – набор для разработки (компилятор, библиотеки и др.).
        \item \textcolor[rgb]{0.0, 0.5, 0.0}{\textbf{JIT (Just In Time)}} – компилятор, ускоряющий выполнение кода в JVM.
        \item \textcolor[rgb]{0.0, 0.5, 0.0}{\textbf{JRE (Java Runtime Environment)}} – среда для запуска Java-программ (JVM + библиотеки).
        \item \textcolor{red}{JEE (Java Enterprise Edition)} – платформа для корпоративных приложений.
        \item \textcolor{red}{J2EE (Java 2 Enterprise Edition)} – старое название JEE.
        \item \textcolor{red}{JSE (Java Standard Edition)} – базовая версия Java для десктопных приложений.
        \item \textcolor{gray}{Garbage collector} – автоматически удаляет неиспользуемые объекты из памяти.
    \end{itemize}
\end{frame}

\section{Среда разработки, тестирование и сборщики проекта}
\begin{frame}{Среда разработки, сборщики проекта и тестирование}
    \begin{itemize}
        \item \textcolor[rgb]{0.0, 0.5, 0.0}{IntelliJ IDEA} (CE / \textcolor[rgb]{0.0, 0.5, 0.0}{Ultimate}) / Eclipse IDE / NetBean - среды разработки
        \item \textcolor[rgb]{0.0, 0.5, 0.0}{Gradle} / Apache Maven - сборщики проектов 
        \item \textcolor[rgb]{0.0, 0.5, 0.0}{JUnit} / Mockito - библиотеки для тестирования
    \end{itemize}
\end{frame}

\section{4 принципа ООП в Java}
\begin{frame}{4 принципа ООП в Java}
\begin{rusdefinition}
    \begin{itemize}
        \item \textcolor{red}{Абстракция} – выделение главных характеристик объекта, игнорируя несущественные детали.
        \item Полиморфизм – возможность объектов с одинаковой спецификацией иметь разную реализацию.
        \item Инкапсуляция – объединение данных и методов в единый объект, скрывая внутреннюю реализацию.
        \item Наследование – создание новых классов на основе существующих с переносом их свойств и методов.
    \end{itemize}
\end{rusdefinition}
\end{frame}

\subsection{Абстракция}
\begin{frame}{Абстракция}
\begin{rusdefinition}
    Упрощение реальности через выделение значимых характеристик объекта. 
\end{rusdefinition}
\begin{rexample}
    \begin{itemize}
        \item Нам неважно как именно работает изнутри самолет: 
        \item мы знаем \textit{что} самолет летит, 
        \item но не \textit{как} он это делает (запуск двигателей, закрытие шасси, включение gps, и т.д.).
    \end{itemize}
\end{rexample}
\end{frame}

\subsection{Инкапсуляция}
\begin{frame}{Инкапсуляция}
\begin{rusdefinition}   
Сокрытие внутренней реализации и защита данных.
\end{rusdefinition}
\begin{rexample}
    \begin{itemize}
        \item Банкомат: 
        \item мы вводим PIN (публичный метод), 
        \item но не можем напрямую изменить баланс на счете, так как это приватное поле. Эх$\dots$
    \end{itemize}
\end{rexample}
\end{frame}

\subsection{Полиморфизм}
\begin{frame}{Полиморфизм}
\begin{rusdefinition}
Разное поведение объектов при одном интерфейсе. 
\end{rusdefinition}
\begin{rexample}
    \begin{itemize}
    \item Физическая красная кнопка "Пуск":
    \item На пульте управления ракеты она запускает ракету,
    \item На пульте управления телевизора — включает телевизор,
    \item Но форма кнопки (интерфейс) одинакова: круглая, красная.
\end{itemize}
\end{rexample}
\end{frame}

\subsection{Наследование}
\begin{frame}{Наследование}
    \begin{rusdefinition}
    Переиспользование свойств родительского класса. 
\end{rusdefinition}
\begin{rexample}
    \begin{itemize}
        \item Транспорт: 
        \item У \textit{Велосипеда} и \textit{Автомобиля} есть общие черты (колеса, движение),
        \item но у каждого — уникальные особенности (мотор, педали).
    \end{itemize}
\end{rexample}
\end{frame}

\subsubsection{Множественное наследование}
\begin{frame}{Множественное наследование}
    \setbeamertemplate{itemize/enumerate body begin}{\scriptsize}
    \setbeamertemplate{itemize/enumerate subbody begin}{\tiny}
    
    \begin{rusdefinition}
        \footnotesize
        Множественное наследование - возможность класса иметь несколько непосредственных родительских классов, наследуя их свойства и методы. \\
        \textbf{\textcolor{red}{В Java множественное наследование запрещено!}}
    \end{rusdefinition}

    \vspace{-0.2cm}
    \begin{rusdefinition}
        \footnotesize
        В Java множественное наследование классов запрещено для:
        \begin{itemize}
            \item Избежания "проблемы ромба" (неоднозначность при вызове методов)
            \item Упрощения архитектуры
        \end{itemize}
    \end{rusdefinition}

    \vspace{-0.2cm}
    \begin{rexample}
        \footnotesize
        Альтернативы в Java:
        \begin{itemize}
            \item \textbf{Интерфейсы} (множественная реализация)
            \item \textbf{Композиция} (включение объектов других классов)
            \item \textbf{Дефолтные методы} (реализация в интерфейсах с Java 8)
        \end{itemize}
    \end{rexample}
    
    \setbeamertemplate{itemize/enumerate body begin}{\normalsize}
    \setbeamertemplate{itemize/enumerate subbody begin}{\normalsize}
\end{frame}

\section{Стиль написания Java-кода}
\begin{frame}{Стиль написания Java-кода}
    \begin{itemize}
        \item UpperCamelCase - классы, файлы классов.
        \item lowerCamelCase - функции, переменные. 
        \item UPPERCASE - перечисления в enum.
        \item lowercase - пакеты.
    \end{itemize}
    \begin{rexample}
        \begin{itemize}
            \item class MyClass - класс
            \item void myFunction - метод (функция)
            \item int myInt - переменная 
            \item enum MyEnum = {MATH, IT} - перечисления 
            \item org.mypackage.lesson - пакет в пакетах
        \end{itemize}
    \end{rexample}
    
\end{frame}

\section{Первая программа}
\begin{frame}{Первая программа}
    \begin{rusdefinition}
        \begin{itemize} 
            \item \textcolor{red}{public class - должен называться так же как файл .java}
            \item \textcolor{red}{public class - должен быть единственным в файле}
        \end{itemize}
    \end{rusdefinition}

    \begin{forMain}
        public class Main\{ \\
            \texttt{ } public static void main(String[] args)\{ \\
            \texttt{ } \texttt{ } System.out.println("Let's start"); \\
            \texttt{ } \}    \\
        \}
    \end{forMain}
\end{frame}

\begin{frame}{}
    \centering
    \Large Спасибо за внимание! 

    \small \textcolor{red}{Пишите вопросы в комментариях!!!}
    
    \vspace{1cm}
\end{frame}

\end{document}