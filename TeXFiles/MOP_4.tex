\documentclass{beamer}
\usepackage[T1,T2A]{fontenc}
\usepackage[utf8]{inputenc}
\usepackage[english,russian]{babel}
\usepackage{amsmath}
\usepackage{graphicx}
\usepackage{listings}
\usepackage{xcolor}

\usetheme{Madrid}
\definecolor{mygreen}{RGB}{0, 128, 0} 

\setbeamercolor{structure}{fg=mygreen} 
\setbeamercolor{palette primary}{fg=white, bg=mygreen} 
\setbeamercolor{titlelike}{fg=white, bg=mygreen} 
\setbeamercolor{block title}{fg=white, bg=mygreen} 

\title[МОП]{Краткая теория множеств}
\author{Салимли Айзек}
\institute{MathLang}
\date{\today}
\newenvironment{rusdefinition}[1][Определение]{
    \begin{block}{#1}
}{\end{block}}
\newenvironment{rexample}[1][Пример]{\begin{exampleblock}{#1}}{\end{exampleblock}}

\lstset{
    language=Haskell,
    basicstyle=\ttfamily\small,
    inputencoding=utf8,
    extendedchars=true,
    literate={�}{{\bfseries?}}1 {^^ff}{{\bfseries?}}1,
    breaklines=true,
    breakatwhitespace=true,
    tabsize=2,
    frame=single
}

\begin{document}

\begin{frame}
    \titlepage
\end{frame}

\AtBeginSection[]{
    \begin{frame}{Содержание}
        \tableofcontents[currentsection]
    \end{frame}
}

\section{Бинарный код}
\begin{frame}{Бинарный код}
\begin{rusdefinition}{Бинарный код}
Способ представления информации с помощью двух символов: 0 и 1. Каждая цифра называется \textbf{битом}.
\end{rusdefinition}

\begin{rexample}{Примеры}
\begin{itemize}
\item Число 5: $101_2$
\item Буква 'A' в ASCII: $01000001_2$
\item Истина/Ложь: $1/0$
\end{itemize}
\end{rexample}

\begin{block}{Математическое представление}
\[
\text{Число} = \sum_{i=0}^{n} b_i \times 2^i \quad \text{где } b_i \in \{0,1\}
\]
\end{block}
\end{frame}

\section{Булевы функции}
\begin{frame}{Булевы функции}
\begin{rusdefinition}{Булева функция}
Функция $f: \{0,1\}^n \to \{0,1\}$, где аргументы и результат принимают значения 0 (Ложь) или 1 (Истина).
\end{rusdefinition}

\begin{rexample}{Базовые функции}
\begin{itemize}
\item Тождество: $f(x) = x$
\item Отрицание: $f(x) = \neg x$
\item Конъюнкция: $f(x,y) = x \land y$
\end{itemize}
\end{rexample}
\end{frame}
\begin{frame}{Таблица истинности для $\neg$}
    \begin{block}{Таблица истинности для $\neg$}
        \begin{centering}
        \begin{tabular}{c|c}
            $x$ & $\neg x$ \\ \hline
            0 & 1 \\
            1 & 0 \\
            \end{tabular}
        \end{centering}
    \end{block}
    \end{frame}
\begin{frame}{Таблица истинности для $\land$}
\begin{block}{Таблица истинности}
    \begin{centering}
\begin{tabular}{cc|c}
x & y & $x \land y$ \\ \hline
0 & 0 & 0 \\
0 & 1 & 0 \\
1 & 0 & 0 \\
1 & 1 & 1 \\
\end{tabular}
\end{centering}
\end{block}
\end{frame}

\begin{frame}{Таблица истинности для $\lor$}
    \begin{block}{Таблица истинности для $\lor$}
        \begin{centering}
    \begin{tabular}{cc|c}
    x & y & $x \lor y$ \\ \hline
    0 & 0 & 0 \\
    0 & 1 & 1 \\
    1 & 0 & 1 \\
    1 & 1 & 1 \\
    \end{tabular}
\end{centering}
    \end{block}
    \end{frame}

    \begin{frame}{Таблица истинности для $\oplus$}
        \begin{block}{Таблица истинности для $\oplus$}
            \begin{centering}
        \begin{tabular}{cc|c}
        x & y & $x \lor y$ \\ \hline
        0 & 0 & 0 \\
        0 & 1 & 1 \\
        1 & 0 & 1 \\
        1 & 1 & 0 \\
        \end{tabular}
    \end{centering}
        \end{block}
        \end{frame}
    

\section{Основные операции}
\begin{frame}{Основные операции}
\begin{rusdefinition}{Логические операции}
\begin{itemize}
\item \textbf{НЕ} ($\neg$): Отрицание
\item \textbf{И} ($\land$): Конъюнкция
\item \textbf{ИЛИ} ($\lor$): Дизъюнкция
\item \textbf{искл. ИЛИ} ($\oplus$): Сложение по модулю 2 
\end{itemize}
\end{rusdefinition}

    \begin{rexample}{Выражения}
\begin{itemize}
\item $\neg(1 \lor 0) = 0$
\item $(1 \land \neg 0) \oplus 1 = 0$
\end{itemize}
\end{rexample}
\end{frame}

\begin{frame}{Основные операции}
\begin{block}{Битовые аналоги}
\begin{itemize}
\item И: \texttt{\&}
\item ИЛИ: \texttt{|}
\item НЕ: \texttt{\~}
\end{itemize}
\end{block}
\end{frame}

\section{Порядок операций}
\begin{frame}{Порядок операций}
\begin{rusdefinition}{Приоритет операций}
\begin{enumerate}
\item Отрицание ($\neg$)
\item Конъюнкция ($\land$)
\item Дизъюнкция ($\lor$)
\item Исключающее ИЛИ ($\oplus$), и др.
\end{enumerate}
\end{rusdefinition}

\begin{rexample}{Пример вычисления}
\[
\neg 1 \lor 0 \land 1 = (\neg 1) \lor (0 \land 1) = 0 \lor 0 = 0
\]
\end{rexample}

\begin{alertblock}{Важно!}
Всегда используйте скобки для явного задания порядка:
$(\neg (1 \lor 0)) \land 1$
\end{alertblock}
\end{frame}
\section{Зачем это нужно?}
\begin{frame}{Зачем это нужно?}
    \textbf{\textcolor{red}{Компьютер понимает лишь 0 и 1, языки программирования, а точнее компилятор переводчик (инструмент абстракции) с ЯП в бинарный код}}
    \[ ProgLang \rightarrow MachineCode \]
    Булевые операции и функции используются в:
    \begin{itemize}
        \item Электронике
        \item Схемы (Logisim, Multisim)
        \item Шифровании 
        \item Криптографии
    \end{itemize}
\end{frame}
\begin{frame}{}
    \centering
    \Large Спасибо за внимание!
    
    \vspace{1cm}
    \small \textcolor{red}{Пишите вопросы в комментариях!!!}
\end{frame}

\end{document}